\documentclass{article}

\usepackage[latin1]{inputenc}
\usepackage{tikz}
\usetikzlibrary{shapes,arrows}

\usepackage{titlesec}

\titlespacing\section{0pt}{12pt plus 4pt minus 2pt}{0pt plus 2pt minus 2pt}

\usepackage[margin=1.5in]{geometry}

\begin{document}
\pagestyle{empty}



\section{Description of Natural System}

Atlantic Herring are a schooling fish which hunt copepods using a technique known as "Ram Feeding". The copepods have an evasive jump that is faster than any individual herring, but the herring school in a grid that corresponds to the typical jump distance of the copepods. The school succeeds at catching copepods where individuals fail.

\section{Research Question}

Are schooling herring more effective at hunting copepods than those that do not school, or herring that school with different grid distances?

\section{Agent types}

\begin{enumerate}
\itemsep 0em 
\item Herring - Position, Direction, Velocity, Acceleration, Sense radius, Herring Attraction, Copepod Attraction, Alignment strength,  Hunger? Fatigue?
\item Copepod - Position, Direction, Velocity, Acceleration, Sense radius, Jump distance/velocity
\end{enumerate}

\section{Functions/Behaviours}

\begin{itemize}
\itemsep 0em 
\item findAgentsInRadius
\item calcHerringVect
\item eatCopepod
\item calcCopepodVect
\item copepodJump

\end{itemize}


\section{Flow diagram}

\tikzstyle{block} = [rectangle, draw, fill=blue!20,
    text width=5em, text centered, rounded corners, minimum height=2em, node distance=2.4cm]
\tikzstyle{line} = [draw, very thick, color=black!50, -latex']

\begin{tikzpicture}[scale=2, node distance = 2cm, auto]
    % Place nodes
    \node [block] (init) {initialize model};
    \node [block, right of=init] (nsteps) {for i = 1:num steps};
    \node [block, below of=nsteps, xshift = 1cm] (nrules) {for m = 1:num rules};
    \node [block, right of=nrules] (rules) {Rules: 1. Sense 2. Move 3. Eat};
    
    \node [block, below of=nrules, xshift = 1cm] (nagents) {for a = 1:num agents};
    \node [block, right of=nagents] (agenttypes) {Shuffled list of copepods + herring};
    
    \node [block, below of=nagents, xshift = 1cm] (agent) {apply rule m to agent a};
    \node [block, left of=agent, xshift = -3cm] (output) {Save/plot output};
   
    \path [line] (init) -- (nsteps);
    \path [line] (nsteps) -- (nrules);
    \path [line] (nrules) -- (nagents);
    \path [line] (nagents) -- (agent);
    \path [line] (agent) -- (output);
    
\end{tikzpicture}

\end{document}